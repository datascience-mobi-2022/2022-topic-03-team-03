% Options for packages loaded elsewhere
\PassOptionsToPackage{unicode}{hyperref}
\PassOptionsToPackage{hyphens}{url}
%
\documentclass[
]{article}
\title{Proteome wide screen for RNA-dependent Proteins in interphasic
HeLa cells}
\author{Bolz, C., Bonsen, M., Pott, M., Simon, M.}
\date{2022-07-17}

\usepackage{amsmath,amssymb}
\usepackage{lmodern}
\usepackage{iftex}
\ifPDFTeX
  \usepackage[T1]{fontenc}
  \usepackage[utf8]{inputenc}
  \usepackage{textcomp} % provide euro and other symbols
\else % if luatex or xetex
  \usepackage{unicode-math}
  \defaultfontfeatures{Scale=MatchLowercase}
  \defaultfontfeatures[\rmfamily]{Ligatures=TeX,Scale=1}
\fi
% Use upquote if available, for straight quotes in verbatim environments
\IfFileExists{upquote.sty}{\usepackage{upquote}}{}
\IfFileExists{microtype.sty}{% use microtype if available
  \usepackage[]{microtype}
  \UseMicrotypeSet[protrusion]{basicmath} % disable protrusion for tt fonts
}{}
\makeatletter
\@ifundefined{KOMAClassName}{% if non-KOMA class
  \IfFileExists{parskip.sty}{%
    \usepackage{parskip}
  }{% else
    \setlength{\parindent}{0pt}
    \setlength{\parskip}{6pt plus 2pt minus 1pt}}
}{% if KOMA class
  \KOMAoptions{parskip=half}}
\makeatother
\usepackage{xcolor}
\IfFileExists{xurl.sty}{\usepackage{xurl}}{} % add URL line breaks if available
\IfFileExists{bookmark.sty}{\usepackage{bookmark}}{\usepackage{hyperref}}
\hypersetup{
  pdftitle={Proteome wide screen for RNA-dependent Proteins in interphasic HeLa cells},
  pdfauthor={Bolz, C., Bonsen, M., Pott, M., Simon, M.},
  hidelinks,
  pdfcreator={LaTeX via pandoc}}
\urlstyle{same} % disable monospaced font for URLs
\usepackage[margin=1in]{geometry}
\usepackage{longtable,booktabs,array}
\usepackage{calc} % for calculating minipage widths
% Correct order of tables after \paragraph or \subparagraph
\usepackage{etoolbox}
\makeatletter
\patchcmd\longtable{\par}{\if@noskipsec\mbox{}\fi\par}{}{}
\makeatother
% Allow footnotes in longtable head/foot
\IfFileExists{footnotehyper.sty}{\usepackage{footnotehyper}}{\usepackage{footnote}}
\makesavenoteenv{longtable}
\usepackage{graphicx}
\makeatletter
\def\maxwidth{\ifdim\Gin@nat@width>\linewidth\linewidth\else\Gin@nat@width\fi}
\def\maxheight{\ifdim\Gin@nat@height>\textheight\textheight\else\Gin@nat@height\fi}
\makeatother
% Scale images if necessary, so that they will not overflow the page
% margins by default, and it is still possible to overwrite the defaults
% using explicit options in \includegraphics[width, height, ...]{}
\setkeys{Gin}{width=\maxwidth,height=\maxheight,keepaspectratio}
% Set default figure placement to htbp
\makeatletter
\def\fps@figure{htbp}
\makeatother
\setlength{\emergencystretch}{3em} % prevent overfull lines
\providecommand{\tightlist}{%
  \setlength{\itemsep}{0pt}\setlength{\parskip}{0pt}}
\setcounter{secnumdepth}{-\maxdimen} % remove section numbering
\ifLuaTeX
  \usepackage{selnolig}  % disable illegal ligatures
\fi

\begin{document}
\maketitle

\hypertarget{introduction}{%
\section{Introduction}\label{introduction}}

RNA and proteins represent a symbiotic system. Proteins need RNA as a
template for their biosynthesis as stated by Francis Crick in his
``Central Dogma of Molecular Biology'' (Crick, 1970). The opposite is
also true, studies have shown that proteins need RNA for their catalytic
activity, for instance in the RNA-induced silencing complex (Pratt and
Macrae, 2009; Wilson and Doudna, 2013). Additionally, some RNA sequences
depend on proteins for their synthesis and stability (Kishore \emph{et
al.}, 2010). A family of proteins which illustrates this symbiotic
relationship are the RNA-binding proteins (RBPs). RBPs present a class
of proteins whose interactome depends on RNA. They were shown to play a
crucial role in RNA metabolism (Kishore \emph{et al.}, 2010), cancer
development (Wei \emph{et al.}, 2022), and genetic disease (Gebauer
\emph{et al.}, 2021). Therefore, a deeper understanding of RNA binding
and RNA-protein interaction strengthens our ability to adjust and
manipulate the cellular mechanisms affected.

RBPs can be categorized into ``true'' RBPs (e.g.~DICER, NPM3), those
RNA-binding proteins which directly bind to RNA, and ``RBP interacting
proteins'' which merely interact with ``true'' RBPs (eg. RBBP7).
Furthermore, ``true'' RBPs can be subcategorized into ``RNA-dependent'',
meaning relying on RNA for their whole and correct function (eg. DICER)
and ``partially RNA dependent'', those that only require RNA for certain
functions or transport (eg. NPM3) (see Fig. 1) (Caudron-Herger \emph{et
al.}, 2019; Corley \emph{et al.}, 2020).

\textbf{Figure 1}. Categorization of RNA-binding proteins: RBPs can be
categorized into ``true'' RBPs, those RNA-binding proteins which
directly bind to RNA, ``RBP interacting proteins'' which merely interact
with ``true'' RBPs. ``True'' RBPs are subcategorized into
``RNA-dependent'', meaning relying on RNA for their whole and correct
function and ``partially RNA dependent'', those that only require RNA
for certain functions or transport. DICER = Endoribonuclease Dicer; NPM3
= Nucleoplasmin 3; RBBP7 = RNA-binding protein binding protein 7.

Several approaches to study RNA-protein interaction and to identify new
RBPs were established such as RaPID and CLIP-Seq. These methods either
analyze the interaction between the RNA of interest and additional
proteins (RaPID) or the interaction between a protein of interest and
the different RNA with which it is interacting (CLIP-Seq) (Qin \emph{et
al.}, 2021). Therefore, approaching the global study of RNA-binding
protein, interaction networks (Sternburg and Karginov, 2020) has become
a matter of interest. A method published by Caudron-Herger \emph{et al.}
enables analysis and quantification of whole cell interactomes.
Furthermore, this method allows for identification of new RBPs through
RNase treatment and density gradient ultracentrifugation. Subsequently,
the resulting fraction shifts of proteins identified via mass
spectrometry were analyzed using bioinformatic techniques
(Caudron-Herger \emph{et al.}, 2019, Caudron-Herger \emph{et al.},
2020).

In this project, we identified RBPs and possible RBP candidates using
bioinformatics in R. Beyond that, we further identified contributing
variables in our data differentiating RBPs without relying on our entire
analysis protocol. We cross-referenced our results with known databases
such as R-DeeP (\url{https://r-deep.dkfz.de/}; Caudron Herger \emph{et
al.}, 2019), UniProt (\url{https://www.uniprot.org/}) and RBP2GO
(\url{https://rbp2go.dkfz.de/}; Caudron-Herger \emph{et al.}, 2021). Our
dataset focusing on interphase synchronized HeLa S3 cells was obtained
by the method published by Caudron-Herger \emph{et al.} cited above.

\newpage

\hypertarget{methods}{%
\section{Methods}\label{methods}}

\hypertarget{generation-of-the-dataset}{%
\subsection{Generation of the dataset}\label{generation-of-the-dataset}}

Interphasic HeLa S3 cells were harvested and lyzed. The lyzate was split
into two samples and one sample was treated with a RNase mixture,
referred to as RNase (``RNA''). The other sample was left untreated,
referred to as a control (``ctrl''). Both samples were loaded onto a 5
\% to 25 \% sucrose density gradient, grouped in 25 fractions and
separated using ultracentrifugation. For either sample, technical
triplicates were generated. Each fraction was then analyzed using mass
spectrometry. Proteins were identified using UniProt. The data for the
amount of protein per fraction, condition and replicate was collected in
arbitrary units and stored in a .csv-file (for full protocol see:
Caudron-Herger \emph{et al.}, 2019).

\hypertarget{clean-up-and-sorting-of-the-dataset}{%
\subsection{Clean-Up and sorting of the
dataset}\label{clean-up-and-sorting-of-the-dataset}}

Using the R package tidyverse, two separate data frames were generated
containing either all untreated (``\_ctrl'') or RNase treated
(``\_RNA'') replicates. Both data frames, as well as the raw data set,
were screened for rows containing zeros only. Those zero rows were
removed from the data frame and stored in a new data frame.

\hypertarget{normalization}{%
\subsection{Normalization}\label{normalization}}

To rule out batch to batch effects and technical error, the protein
amount per replicate (``Rep'') and fraction (``Frac'') were additionally
normalized with respect to each sample.The normalized results were then
visualized by plotting the protein distribution in both samples.

\hypertarget{fraction-wise-normalization}{%
\subsubsection{Fraction-wise
normalization}\label{fraction-wise-normalization}}

Normalization for each fraction was performed applying the following
equation:

\[Protein(norm) = \frac{maxcolsums}{ColSumme} * Protein(before)\]

\emph{Protein(norm)} describes the normalized amount of a single protein
per fraction and replicate.\emph{ColSumme} represents the total amount
of protein per fraction and replicate. For each fraction, one maximum
was chosen. \emph{maxcolsums} is the selected maximum total protein
amount per fraction and replicate. The parameter \emph{Protein(before)}
describes the arbitrary amount of a single protein per fraction and
replicate prior to normalization.

\hypertarget{scaling-the-protein-distrubition-per-replicate-and-condition}{%
\subsubsection{Scaling the protein distrubition per replicate and
condition}\label{scaling-the-protein-distrubition-per-replicate-and-condition}}

For better comparison, the protein amount per fraction was converted to
a relative percentage scale applying the following equation:

\[Protein(relative) = \frac{Protein(norm)}{RowSum}*100\]

\emph{Protein(relative)} describes the relative protein amount per
fraction in relation to the total amount of the protein per replicate.
\emph{Protein(norm)} represents the normalized amount of a single
protein per fraction and replicate. The total amount of a single protein
per replicate is given by \emph{RowSum}.

\hypertarget{create-separate-dataframes}{%
\subsubsection{Create separate
dataframes}\label{create-separate-dataframes}}

Next, data frames for each individual fraction and condition were
created, containing the triplicate values for each protein.

They follow the naming convention: 1. Fraction1\_Ctrl for the triplicate
values of the first fraction of the control group 2. Fraction\_8\_RNAse
for the triplicate values of the eighth fraction of the RNase treated
sample

\hypertarget{calculation-of-means-and-standard-deviation}{%
\subsection{Calculation of means and standard
deviation}\label{calculation-of-means-and-standard-deviation}}

The mean and standard deviation of the triplicates for each protein in
each fraction were calculated using the built-in functions
\emph{mean()-}(\(\overline x\)) and \emph{sd()-function}\((\sigma)\) of
R. Outliers were detected using \(\overline x \pm 3\sigma\) as a
cut-off. All values below and above the three sigma cut-off were
replaced with NA and not considered in the following calculations.

\hypertarget{shapiro-wilk-test}{%
\subsection{Shapiro-Wilk Test}\label{shapiro-wilk-test}}

To test the normal distribution of the triplicate values, the
Shapiro-Wilk test was chosen due to its high power in small populations
compared to other tests. The test was performed using the built-in R
function \emph{shapiro\_test()}.

\hypertarget{determination-of-maxima}{%
\subsection{Determination of Maxima}\label{determination-of-maxima}}

We determined the global maximum for the RNase-treated and untreated
sample for all proteins. The protein content (y-value) was compared to
the two neighbors right and left of the analyzed fraction (x-position).
For fractions 1 and 25 only the neighbors right or left of the fraction
could be compared due to border limitations. For fractions 2 and 24 only
one neighbor could be compared left or right. Obtained values were
stored in a sperate data frame.

\hypertarget{detection-of-protein-shifts}{%
\subsection{Detection of Protein
Shifts}\label{detection-of-protein-shifts}}

In our analysis, we considered shifts in the global maximum comparing
RNA-treated and untreated samples as a proxy for the presence of RNA. We
required ``shifting proteins'' show both a significant x-shift and
y-shift to improve precision.

\hypertarget{x-shifts}{%
\subsubsection{x-shifts}\label{x-shifts}}

First, we compared the x-position of the global maximum in both samples.
To quantify the shift, we determined the difference in the fraction
number of the control group and the RNase treated sample.

\[x-shift = |fracs\_max\_ctrl| - |fracs\_max\_rnase|\]

The variables \emph{fracs\_max\_ctrl} and \emph{fracs\_max\_rnase}
describe the x-position of the global maxima for control and RNase
treatment respectively. \emph{x-shift} is the resulting value and used
for Shift-direction determination.

The following convention regarding x-shifts was used:

\begin{enumerate}
\def\labelenumi{\arabic{enumi}.}
\item
  Left-shift:\tab    Value \textless{} 1
\item
  Right-shift:\tab   Value \textgreater{} 1
\item
  No-Shift:\tab      Value = 0
\end{enumerate}

\hypertarget{y-shifts}{%
\subsubsection{y-shifts}\label{y-shifts}}

The total value of the y-shift was calculated as the difference between
the y-values of the global maxima for both conditions.

\[y-shift\_total = |absolute\_max\_ctrl| - |absolute\_max\_rnase|\]

The variables \emph{absolute\_max\_ctrl} and \emph{absolute\_max\_rnase}
describe the y-value of the global maxima for control and RNase
treatment respectively. \emph{y-shift\_total} is the resulting value.

\hypertarget{statistical-analysis}{%
\subsection{Statistical analysis}\label{statistical-analysis}}

To identify significant y-shifts, we used statistical analysis to
determine the difference in the relative protein amount (y-value) for
each protein in the global maximum fraction of RNase-treated and
untreated samples. All proteins with x-shift values \(\geqq |1|\) were
considered.

\hypertarget{f-test}{%
\subsubsection{F-Test}\label{f-test}}

To determine whether our sample variances were comparable and therefore
suitable for the two-tailed, unpaired t-test, a two-sided F-Test was
performed on the triplicates of each protein and fraction using the
built-in R function \emph{var.test()}. The significance level (p-value)
was set to α = 0.01. The test was deemed positive if \(p-value > 0.01\).

If \emph{var.test()} was performed on all zero samples, NA/NaN was
returned. Therefore, the F-test filtered out those samples that were not
relevant for maxima analysis anyways. Since only the comparison of
y-values at the x-positions of the global maxima mattered for our
further analysis, proteins that failed the F-test (p-value \textless{}
0.01) at those x-positions were excluded.To quantifiy how many proteins
failed F-test RNAse- and ctrl-maximum spots were analyzed. To refer the
F-test results back to the actual proteins, a more visual matrix was
created labeld \emph{p\_value\_matrix}.

\hypertarget{students-t-test}{%
\subsubsection{Students T-Test}\label{students-t-test}}

The two-tailed, unpaired t-test was used to identify proteins with
significant y-value changes in global maxima x-positions after
treatment. P-Values were calculated for each triplicate per protein per
fraction using the built-in R function \emph{t.test()}. Fractions with a
global maximum were compared to the significance level α = 0.05. The
test was deemed positive if \(p-value < 0.05\). Therefore, the change
was considered significant.

Since the t-test was performed on y-values of x-positions of global
maxima for both conditions separately, the returned results were either
TRUE/TRUE, TRUE/FALSE, FALSE/TRUE or FALSE/FALSE for the ctrl and RNAse
maxima. Only proteins with positive t-tests (p-value \textless{} 0.01)
for the global maximum in both conditions (TRUE/TRUE) were considered
for further analysis.

\hypertarget{identification-of-potential-rbps-by-analysis-of-x--and-y-shift}{%
\subsection{Identification of potential RBPs by analysis of x- and
y-shift}\label{identification-of-potential-rbps-by-analysis-of-x--and-y-shift}}

Potential RBPs were selected by filtering out proteins with significant
y-shift following the t-test but no significant x-shift in the global
maxima. Two different cut-off conditions were tested.

\begin{enumerate}
\def\labelenumi{\arabic{enumi}.}
\item
  significant y-shift and x-shift \textless{} 1
\item
  significant y-shift and x-shift \textless{} 2
\end{enumerate}

Proteins filtered out by this method were removed from the data frame
and stored separately for further analysis.

\hypertarget{checking-for-false-positive-and-false-negative-results}{%
\subsubsection{Checking for false positive and false negative
results}\label{checking-for-false-positive-and-false-negative-results}}

False-positive and false-negative proteins were determined by
cross-referencing with two data sets which were provided by Maiwen
Caudron-Herger at Prof.~Dr.~Sven Diederichs Lab (DFKZ) (unpublished
Data). The data contains information on RBPs and non-RBPs previously
identified by different researchers.

\hypertarget{identification-of-potential-rbd-proteins}{%
\subsection{Identification of potential RBD
Proteins}\label{identification-of-potential-rbd-proteins}}

Potential RNA-binding dependent (RBD) protein candidates were identified
by cross-referencing false-positive proteins with the data set provided
by Maiwen Caudron-Herger at Prof.~Dr.~Sven Diederichs Lab (DFKZ)
(unpublished Data).

\hypertarget{k-means-clustering}{%
\subsection{k-means clustering}\label{k-means-clustering}}

Using the libraries \emph{corrplot}, \emph{cluster} and
\emph{factoextra} k-means clustering was conducted using k-means
algorithm. For clustering the follwing variables were selected:

-quotient function: RNAse-value/ctrl-value (high values if it's a
shifter) -calculated rowSums of each proteins quotient (25 each)
-cluster variables: q\_mat\_sums (rowsum) and maximum quotient value
-best within cluster square distance with 3 clusters (although
silhouette method said 2) - insert cluster diagramm (around (0,0) non
shifters -\textgreater{} small rowsum and small maximum, on the top
right -\textgreater{} shifters as they have big maximum and rowsum) and
in cluster between those 2 maybe proteins that don't shift as clearly as
cluster 3.

\hypertarget{comparing-clustering-results}{%
\subsection{Comparing clustering
results}\label{comparing-clustering-results}}

\hypertarget{linear-regression-analysis}{%
\subsection{Linear regression
analysis}\label{linear-regression-analysis}}

\newpage

\hypertarget{results}{%
\section{Results}\label{results}}

\hypertarget{clean-up-and-normalization}{%
\subsection{Clean-Up and
normalization}\label{clean-up-and-normalization}}

The initial data set consisted of 7086 proteins. After clean-up 7081
proteins remained for further analysis. Normalization was visualized for
several reported sure-shifters and non-shifters (see Caudron-Herger
\emph{et al.}, 2019). The selected proteins are:

Sure-shifters: Sin3A\_HUMAN, HDAC1\_HUMAN, HNRPU\_HUMAN, RFC2\_HUMAN

Non-shifters: ASNS\_HUMAN, MCM2\_HUMAN, MCM3\_HUMAN

Through normalization the arbitrary values were converted into relative
percentage scale. Yet, the overall distribution of the protein amount
per fraction remained comparable to the distribution prior to
normalization. Therefore, the normalization worked for sure-shifters as
well as non-shifters. For instance, figure 2 displays the normalization
of a sure-shifter (RFC2\_HUMAN) and a non-shifter (MCM2\_HUMAN).

\textbf{Figure 2: Comparing pre and post normalization} (A): Comparison
of the sure-shifter RFC2\_HUMAN prior to and post normalization. (B):
Comparison of the non-shifter MCM3\_HUMAN prior to and post
normalization. For both selected proteins, the overall trend of
distribution remains comparable.

\hypertarget{calculation-of-mean-sd-and-shapiro-wilk-test}{%
\subsection{Calculation of mean, sd and Shapiro-Wilk
Test}\label{calculation-of-mean-sd-and-shapiro-wilk-test}}

With the proposed 3\(\sigma\) rule we were able to detect not a single
protein in which at least one replicate was not congruent with a normal
distribution. To verify this result Shapiro-Wilk testwas conducted. Test
concluded that all values obtained for each protein were normally
distributed for both conditions. Therefore they were suitable for
statistical analysis with F-Test and two-tailed unpaired t-test.

\hypertarget{statisitcal-analysis-and-identification-of-potential-rbps-by-analysis-of-x--and-y-shift}{%
\subsection{Statisitcal analysis and Identification of potential RBPs by
analysis of x- and
y-shift}\label{statisitcal-analysis-and-identification-of-potential-rbps-by-analysis-of-x--and-y-shift}}

The obtained x- and y-shift as well as the fraction of the global
maxima, shift direction and y-shift values were stored in a data frame
labeled \emph{abs\_max\_both}.

7081 proteins were tested with F-Test. Only the obtained p-values for
fractions were a global maximum was detected were deemed relevant for
identification of potential RBP candidates. Of 7081 proteins 1196 failed
the test and were excluded from further analysis. 5885 remained.
Quantification of FALSE Results in F-Test resulted that 10 \% of all
analyzed proteins fail F-Test. This means that either one or both
obtained maxima for triplicate measurements vary in their variance in
one treatment compared to the other. Because it seems like some FALSE
F-test results originate from outlier values (see Discussion) and
otherwise we would experience a great exclusion of data, significance
level was set to α = 0.01 instead of the previously usual α = 0.05. With
α = 0.01 closer to 5 \% of data will be excluded based on a FALSE F-test
result. In this compromise some more likely faulty data will still be
excluded while retaining as much data for further analysis as logically
arguable.

Two-tailed unpaired t-test was applied to the remaining 5885 proteins
per fraction ctrl vs RNase. P-Values were checked for significance at
previous determined maxima positions. We identified 2929 potential
shifters.

With the defined cut-off parameters significant y-shift and x-shift
\textgreater{} 1 or significant y-shift and x-shift \textgreater{} 2
these potential shifters were further analysed.

The cut-off \textgreater1 for the x-shift allowed us to identify 513
true RBP with 120 false-positives out of 636 total. Of the 120
false-positives six were identified with cross-recferencing as RBD
proteins. In comparison, the cut-off \textgreater2 for the x-shift
allowed us to identify 398 true RBP with 91 false-positives out of 489
total. Of the 91 false-positives six were identified with
cross-referencing as RBD proteins. The six identified proteins were
identical for both cut-off parameters (see Tab.1, data frames
\emph{RBD\_true\_port\_df1} and \emph{RBD\_true\_prot\_df2}).

In both cases, about 80\% of identified RBP were correct when
cross-referenced. Since the \textgreater1 x-shift cut-off did not worsen
this quote but instead delivered more results overall (636 compared to
489), \textgreater1 x-shift cut-off is used for our analysis.

\textbf{Table.1: Amount of identified RBPs:} True RBP quantification
results after cross-referencing obtained proteins with the additional
data provided by Caudron-Herger. TP = True positive, TN = True negative,
FP = False positive, FN = False negative.

\begin{longtable}[]{@{}l@{}}
\toprule
\endhead
Criteria \textbar{} x-Shift \textgreater{} 1 \textbar{} x-shift
\textgreater{} 2 \textbar{} \\
Total identified \textbar{} 636 \textbar{} 489 \textbar{} \\
TP \textbar{} 513 \textbar{} 398 \textbar{} \\
FP \textbar{} 120\textbar{} 91 \textbar{} \\
RBD\textbar{} 6 \textbar{} 6\textbar{} \\
\bottomrule
\end{longtable}

\hypertarget{k-means-clustering-and-regression-model}{%
\subsection{K-Means Clustering and Regression
model}\label{k-means-clustering-and-regression-model}}

\hypertarget{discussion}{%
\section{Discussion}\label{discussion}}

Lorem ipsum

\end{document}
